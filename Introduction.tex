\chapter*{Introduction générale}
\markboth{Introduction générale}{} %pour afficher l'entete
\addcontentsline{toc}{chapter}{Introduction générale}
\large {

\setlength{\parskip}{1em}
\setlength{\parindent}{1cm}

L'intégration des objets connectés dans les systèmes d’information des entreprises est une nécessité pour répondre aux exigences accrues du marché et à l’évolution incessante de la technologie \cite{antoine2019vers}. Dans ce contexte, ce rapport de Mémoire de Mastère porte sur le déploiement de l'infrastructure IT, le développement d'un système d'information et l'intégration des objets connectés au système d’information de l’entreprise Zeta Engineering.  Nous avons divisé ces parties en quatre chapitres pour expliquer notre approche et notre travail :

\begin{itemize}
\item Le premier chapitre présente le projet et son contexte, ainsi que l'organisme d'accueil. Nous expliquons les activités de l'entreprise, ses objectifs et ses besoins en matière d'infrastructure. 
\item Le deuxième chapitre décrit le déploiement de l'infrastructure IT. Nous détaillons les différents éléments nécessaires à intégrer dans une entreprise. 
\item Le troisième chapitre explique le déploiement du système d'information. Nous présentons les choix technologiques que nous avons faits, les différentes étapes de la mise en place du système et les tests effectués pour valider son fonctionnement. 
\item Le dernier chapitre, se concentre sur l'intégration des objets connectés. Nous expliquons comment nous avons relié les objets connectés au système d'information pour collecter et analyser les données collectées. Nous discutons également des défis rencontrés et des solutions mises en place pour y faire face.
\end{itemize}

En conclusion de cette introduction générale, nous avons présenté l'importance croissante de l'intégration des objets connectés dans les systèmes d’information des entreprises. Ce rapport se concentrera sur notre expérience avec l'entreprise Zeta Engineering, détaillant les étapes clés de notre projet.



}