\chapter{Intégration des objets connectés}
%\markboth{Chapitre 4}{Intégration des objets connectés} %pour afficher l'entete
% \addcontentsline{toc}{chapter}{Chapitre 4 : Intégration des objets connectés}

\section{Introduction}

L'intégration des objets connectés est un élément clé pour une infrastructure informatique moderne et efficace. Dans ce dernier chapitre, nous allons explorer les différentes étapes de l'intégration des objets connectés dans notre infrastructure existante. 

Les objets connectés sont des dispositifs électroniques qui peuvent communiquer entre eux et avec des systèmes informatiques, et sont capables de collecter et d'analyser des données en temps réel. 

Ils jouent un rôle important dans la création de systèmes intelligents et autonomes, qui peuvent améliorer l'efficacité, la sécurité et la qualité de vie dans de nombreux domaines. 

Dans ce chapitre, nous examinons les différentes technologies et protocoles utilisés pour l'intégration des objets connectés, ainsi que les défis et les opportunités qu'elle présente. 

Enfin, nous décrivons également les différentes étapes du processus d'intégration, de la sélection des dispositifs appropriés à leur installation, leur configuration et leur maintenance. 



\section{Définition des objets connectés}

Les objets connectés, également connus sous le nom d'internet des objets (IoT), font référence à des appareils physiques qui sont connectés à Internet et peuvent échanger des données avec d'autres appareils ou systèmes \cite{gazis2021iot}.

Les objets connectés sont souvent équipés de capteurs, de logiciels et d'autres technologies permettant de collecter et de communiquer des données en temps réel. Ces données peuvent être utilisées pour améliorer l'efficacité, la productivité et la sécurité dans un large éventail de domaines, tels que la santé, les transports, l'agriculture, l'industrie manufacturière, les villes intelligentes, les maisons intelligentes, etc.

La figure \ref{Chap4.2.1} est un exemple de topologie IoT extraite de l'article "Transfert de données IoT basé sur CoAP d'un Raspberry Pi vers le Cloud" \cite{Scott2019CoAPBI}. Elle présente la topologie d'un système conçu par Thomas Scott et Amna Eleyan.

Ce système se compose de quatre éléments principaux : le capteur, le Raspberry Pi, une passerelle basée sur le protocole CoAP (Constrained Application Protocol) et la plateforme cloud. Le capteur collecte les données et les transmet au Raspberry Pi. Le Raspberry Pi est ensuite responsable de la manipulation des données dans un format adapté pour la transmission via le protocole CoAP. L'implémentation du protocole CoAP communique avec la plateforme cloud, laquelle stocke les données, permettant ainsi l'accès aux utilisateurs.

\begin{figure}[H]
\centering
\includegraphics[width=15cm]{Images/IoT-Topo2.png}
\caption{Architecture IoT basée sur CoAP}
\label{Chap4.2.1}
\end{figure}


L'intégration des objets connectés (IoT) dans les foyers a connu une croissance significative ces dernières années. Selon Statista, en 2020, le nombre d'appareils IoT connectés dans le monde a atteint 26,66 milliards, et ce chiffre devrait augmenter à 75,44 milliards d'ici 2025 \cite{statista-iot-2025}. Les dispositifs IoT incluent une variété d'objets connectés tels que les caméras de surveillance, les machines de pointage, et les lampes intelligentes (smart blubs). 

Les smart blubs sont l'une des applications les plus courantes de l'IoT dans les foyers. Selon un rapport de Grand View Research, le marché des lampes intelligentes devrait atteindre une valeur de 38,68 milliards de dollars d'ici 2027 \cite{market-splash-automation}, avec un taux de croissance annuel composé de 20,8\%. Bien que la plupart des lampes ne soient pas des smart blubs, elles peuvent être connectées à notre réseau grâce à des cartes IoT comme Raspberry Pi ou Arduino.


\section{Intégration des dispositifs de l'entreprise}

Dans cette section, nous explorons l'intégration des dispositifs au sein de notre entreprise. Nous discutons des défis que nous rencontrons dans cette démarche et des solutions que nous envisageons pour les surmonter.

\subsection{Enjeux de l'intégration}

L'intégration des dispositifs d'entreprise constitue un défi essentiel à relever pour améliorer notre infrastructure. Nous devons connecter divers dispositifs qui étaient auparavant isolés au réseau de l'entreprise. Les principaux dispositifs que nous traitons sont :


\begin{itemize}
\item Caméras de surveillance

\item Système de pointage

\item Lampes d'éclairage

\item Capteur de température et d'humidité
\end{itemize}

Chacun de ces dispositifs présente des exigences spécifiques en termes d'intégration et de connectivité. Leur incorporation dans notre réseau doit être soigneusement planifiée pour garantir une opération harmonieuse.

\subsection{Intégration des caméras de surveillance}

Notre première étape consiste à connecter nos caméras de surveillance au réseau de l'entreprise. Pour ce faire, nous configurons chaque caméra pour lui attribuer une adresse IP unique, lui permettant ainsi d'être reconnue sur le réseau. Parallèlement, nous allons utiliser un NVR (Network Video Recorder) pour stocker en continu les enregistrements vidéo et gérer les flux issus des caméras. De plus, nous installons le logiciel SmartPSS sur les ordinateurs des directeurs pour faciliter l'accès aux flux vidéo des caméras.

\begin{figure}[H]
\centering
\includegraphics[width=15cm]{Images/SmartPSS1.png}
\caption{Logiciel SmartPSS installé}
\label{Chap4.2.4}
\end{figure}
\smallskip

SmartPSS \ref{Chap4.2.4} est une solution logicielle développée par Dahua, une entreprise leader dans le domaine des systèmes de vidéosurveillance. Il offre une gestion centralisée des caméras, ainsi que des fonctionnalités d'enregistrement et de lecture des vidéos enregistrées. 


\subsection{Intégration des systèmes de pointage}

L'intégration des systèmes de pointage est une autre composante cruciale de notre initiative. Nous utilisons le logiciel ZKTime \ref{Chap4.2.5}, développé par ZKTeco, pour connecter et contrôler ces systèmes au sein de notre réseau d'entreprise. Ce logiciel nous assurons une communication fluide entre les machines de pointage et le réseau, garantissant ainsi un accès facile et sécurisé aux données de pointage des employés.

\begin{figure}[H]
\centering
\includegraphics[height=10cm,width=4cm]{Images/BRades-Pointage.jpg}\includegraphics[width=12cm]{Images/ZKTime.png}
\caption{Machine de pointage ZKTeco et logiciel ZKTime}
\label{Chap4.2.5}
\end{figure}
\smallskip

\subsection{Gestion intelligente de l'éclairage}

Pour répondre à la demande croissante d'un SmartOffice, ce qui nécessite la mise en place d'un système de contrôle à distance des lampes, nous utilisons une carte IoT, telle que Raspberry Pi ou Arduino, pour connecter les lampes à notre réseau d'entreprise. Cette configuration nous permet une gestion centralisée de l'éclairage, contribuant ainsi à optimiser l'utilisation de l'énergie et à réaliser des économies considérables sur les coûts énergétiques. En fin de compte, l'intégration de dispositifs connectés, tels que des lampes intelligentes, dans notre infrastructure d'entreprise, peut considérablement améliorer l'efficacité opérationnelle et la qualité de vie de nos employés.

\subsection{Gestion intelligente du capteur de température et humidité}

De manière similaire à notre approche pour l'éclairage, nous cherchons à mettre en place une gestion intelligente de notre capteur de température et humidité modèle DHT11. Cela nous permettra de surveiller et de contrôler les conditions thermiques de nos locaux de manière efficace. Nous envisageons d'utiliser Raspberry Pi ou Arduino pour connecter ce capteur à notre réseau. Cette intégration nous permettra de collecter en temps réel des données sur la température et de prendre des mesures proactives pour maintenir des conditions optimales dans nos installations.


\subsection{Solution proposée}

Pour relever ces défis d'intégration et connecter nos dispositifs non-connectés, tels que les lampes et le capteur de température et humidité, à notre réseau, nous proposons d'utiliser une approche basée sur Raspberry Pi en combinaison avec Node-RED. Cette solution offre une méthode cohérente et efficace pour intégrer ces dispositifs dans notre infrastructure existante, tout en fournissant une interface de contrôle conviviale pour la gestion et le suivi de ces dispositifs au sein de notre entreprise.




\section{Solution avec Raspberry Pi et Node-RED}

Dans cette section, nous explorons en détail la solution que nous proposons pour l'intégration de dispositifs au sein de notre entreprise en utilisant Raspberry Pi et Node-RED. Cette approche offre une méthode cohérente et efficace pour connecter nos dispositifs qui ne sont pas nativement connectés, notamment les lampes et le capteur de température, à notre réseau tout en fournissant une interface de contrôle conviviale pour la gestion et le suivi de ces dispositifs au sein de notre entreprise.



\subsection{Matériels utilisés pour notre prototype}

Pour créer notre prototype, nous commençons par présenter les matériels utilisés, comme indiqué dans le tableau \ref{table:matériel-prototype}. Ce prototype sera ultérieurement mis en œuvre sur le bureau principal, puis sur les deux autres bureaux respectivement.


\begin{table}[H]
\begin{center}
\begin{tabular}{|c{3cm}|l{10cm}|}
\hline
\textbf{Image de l'équipement} & \begin{center} \textbf{Description de l'équipement} \end{center} \\
\hline
\includegraphics[width=3cm]{Images/RaspberryPi3.png} & Raspberry Pi B+ V1.2 : Un ordinateur monocarte (single-board computer) de la taille d'une carte de crédit, développé par la fondation Raspberry Pi. Il est équipé d'un processeur ARM Cortex-A53 quad-core à 1,4 GHz, de 1 Go de RAM, de plusieurs ports USB, d'un port Ethernet, de connecteurs audio et vidéo, et d'un slot pour carte microSD pour le stockage. Il peut être utilisé pour une grande variété de projets, notamment des serveurs, des media centers, des stations de développement, des robots, des systèmes embarqués, etc. \\
\hline
\includegraphics[width=3cm]{Images/DHT11.png} & Capteur de température et d'humidité DHT11 : Un composant électronique capable de mesurer la température et l'humidité de son environnement. Il fournit des données essentielles pour notre système de contrôle environnemental. \\
\hline
\includegraphics[width=3cm]{Images/DiodeLED.png} & Diode LED : Une diode électroluminescente utilisée pour simuler le fonctionnement de nos lampes. Elle permet de tester notre système de contrôle d'éclairage. \\
\hline
\end{tabular}
\caption{Les équipements composants de notre prototype miniature}
\label{table:matériel-prototype}
\end{center}
\end{table}




\subsection{Choix de Raspberry Pi et Node-RED}

Face à ce défi, notre choix se porte sur l'utilisation de Raspberry Pi pour connecter les lampes à notre réseau. Raspberry Pi est une carte informatique très performante et polyvalente qui peut être utilisée pour de nombreux projets IoT \cite{richardson2012getting}.

Nous décidons d'utiliser Node-RED, un outil de programmation visuel pour créer une application permettant de contrôler les lampes à distance. Node-RED est une plateforme open source qui facilite la création de flux de données entre différents objets connectés \cite{lekic2018iot}.

En utilisant Raspberry Pi et Node-RED, nous sommes en mesure de créer une solution robuste et évolutive pour connecter les lampes et le capteur de température à notre réseau. Cette solution nous permet de contrôler les lampes à distance à partir de n'importe quel appareil connecté à notre réseau.

\subsection{Installation Raspberry Pi OS et Node-RED}

\subsubsection{a. Installation de l'OS Raspberry Pi}

Pour commencer, nous installons l'OS Debian spécifique au Raspberry Pi sur une carte mémoire. Nous insérons cette carte mémoire ensuite dans notre carte électronique Raspberry Pi pour démarrer le système.

Après avoir téléchargé le Raspberry Pi Imager depuis le site officiel, nous utilisons cet outil pour graver sur notre carte mémoire.
 

Une fois l'image gravée sur la carte mémoire, nous insérons celle-ci dans le slot prévu sur notre carte électronique Raspberry Pi. Cela nous permet de démarrer le Raspberry Pi et d'accéder à l'interface utilisateur du système d'exploitation.

Une fois le Raspberry Pi OS démarré, nous pouvons effectuer les configurations supplémentaires nécessaires pour intégrer les objets connectés à notre infrastructure. Ces configurations peuvent inclure la connexion au réseau, l'installation de logiciels supplémentaires, et la personnalisation des paramètres en fonction de nos besoins spécifiques.

En résumé, l'installation du Raspberry Pi OS sur une carte mémoire et son insertion dans notre carte électronique Raspberry Pi constituent la première étape pour mettre en place notre infrastructure et intégrer les objets connectés. Cette étape est cruciale pour assurer la compatibilité et la stabilité de notre système.

\subsubsection{b. Installation de Node-RED}

Après avoir préparé le Raspberry Pi, nous procédons à l'installation de Node-RED sur la carte à l'aide du script bash officiel disponible sur le site de Node-RED, comme illustré dans la figure \ref{Chap4.3.2}.

\begin{figure}[H]
 \centering
    \includegraphics[width=15cm]{Images/NodeRedInstall1.png}
    \caption{Installation du Node-RED avec le Script Officiel Bash}
    \label{Chap4.3.2}
\end{figure}    

Ce script effectue une série d'étapes, telles que l'arrêt de Node-RED, la suppression des anciennes versions de Node-RED et de Node.js, l'installation de Node.js pour Armv6, la purification du cache npm, l'installation de Node-RED Core, le déplacement des nœuds globaux vers un emplacement local, la reconstruction des nœuds existants, l'installation de nœuds supplémentaires spécifiques à Raspberry Pi, l'ajout de commandes raccourcies et la mise à jour du script systemd comme la montre la figure \ref{Chap4.3.3}.

\begin{figure}[H]
 \centering
    \includegraphics[width=13cm]{Images/NodeRedInstall2.png}
    \caption{En cours d'Installation du Node-RED}
    \label{Chap4.3.3}
\end{figure}    

Une fois l'installation terminée, nous pouvons démarrer Node-RED en utilisant la commande "node-red-start". Le résultat final du démarrage est illustré dans la figure \ref{Chap4.3.5}.

\begin{figure}[H]
 \centering
    \includegraphics[width=15cm]{Images/NodeRedStart1.png}
    \caption{Le démarrage du Node-RED}
    \label{Chap4.3.5}
\end{figure}    

Node-RED est maintenant installé et opérationnel sur notre Raspberry Pi, prêt à être utilisé pour créer des flux et interagir avec les objets connectés de notre infrastructure.


\subsection{Configuration de Node-RED}

Pour accéder à Node-RED via le réseau, nous utilisons l'adresse IP de notre Raspberry Pi et le port 1880/ui.

\begin{figure}[H]
\centering
\includegraphics[width=15cm]{Images/NodeRedInterface.png}
\caption{Interface Node-RED}
\label{Chap4.3.6}
\end{figure}

Dans la figure \ref{Chap4.3.6}, nous observons l'interface de Node-RED, où nous développons notre projet en utilisant des nœuds et des connexions pour créer des flux de données.

Pour notre projet, nous devons installer quelques nœuds supplémentaires. Par exemple, le "dashboard" pour créer une interface graphique conviviale permettant d'interagir avec le projet, et le nœud DHT11 pour détecter notre capteur de température dans Node-RED.

\begin{figure}[H]
\centering
\includegraphics[width=15cm]{Images/Node-Red-Dashboard-Installed.png}
\caption{Installation du dashboard Node-RED}
\label{Chap4.3.7}
\end{figure}

La figure \ref{Chap4.3.7} montre le processus d'installation du nœud "dashboard" dans Node-RED, qui permet de créer des tableaux de bord interactifs pour contrôler et visualiser les données de notre projet.

\begin{figure}[H]
\centering
\includegraphics[width=15cm]{Images/DHT11-Node-Red-Install.png}
\caption{Installation du nœud DHT11 dans Node-RED}
\label{Chap4.3.8}
\end{figure}

Dans la figure \ref{Chap4.3.8}, nous installons le nœud DHT11 dans Node-RED, qui nous permet de capturer les données de notre capteur de température et d'humidité.

Ces différentes figures illustrent la préparation de Node-RED et l'installation des nœuds nécessaires pour notre projet. Elles servent de base à la réalisation de notre système de surveillance et de contrôle.


\subsection{Sécurité de Node-RED}

Pour sécuriser Node-RED, nous utilisons une méthode fournie par le site officiel de Node-RED, qui nous permet de créer des mots de passe hachés et de les ajouter à la configuration de notre projet. Les figures suivantes illustrent ce processus.

\begin{figure}[H]
\centering
\includegraphics[width=14cm]{Images/Node-2.png}
\caption{Configuration de la sécurité dans Node-RED en Settings.js}
\label{Chap4.3.9}
\end{figure}

Dans la figure \ref{Chap4.3.9}, nous modifions le fichier de configuration "Settings.js" de Node-RED pour inclure des paramètres de sécurité. Cela permet de définir des utilisateurs et des mots de passe pour restreindre l'accès à notre projet.

Nous utilisons la commande "node-red admin hash-pw" pour générer un code de hachage sécurisé, comme le montre la figure \ref{Chap4.3.10}.

\begin{figure}[H]
\centering
\includegraphics[width=15cm]{Images/Node-3.png}
\caption{Création d'un code de hachage}
\label{Chap4.3.10}
\end{figure}

Le code de hachage généré est ensuite ajouté au fichier "Settings.js" pour mettre à jour les informations de sécurité, comme indiqué dans la figure \ref{Chap4.3.11}.

\begin{figure}[H]
\centering
\includegraphics[width=15cm]{Images/Node-4.png}
\caption{Mise à jour du code de hachage dans Settings.js}
\label{Chap4.3.11}
\end{figure}

Une fois la sécurité configurée, nous pouvons nous connecter à Node-RED en utilisant le nouveau mot de passe, comme le montre la figure \ref{Chap4.3.12}.

\begin{figure}[H]
\centering
\includegraphics[width=15cm]{Images/Node-5.png}
\caption{Connexion avec le nouveau mot de passe}
\label{Chap4.3.12}
\end{figure}

Ces différentes figures décrivent le processus de sécurisation de Node-RED en utilisant des mots de passe hachés. Cela garantit que seuls les utilisateurs autorisés peuvent accéder à notre projet et protège ainsi notre système contre les accès non autorisés.

\subsection{Réalisation de projet avec Node-RED}

Dans notre topologie, nous utilisons 2 LED, un capteur DHT11 et une breadboard, que nous connectons aux GPIO de Raspberry Pi, comme le montre la figure \ref{Chap4.3.13}, réalisée à l'aide de Fritzing.

\begin{figure}[H]
\centering
\includegraphics[width=15cm]{Images/RaspberryVisual.png}
\caption{Schéma de notre réalisation}
\label{Chap4.3.13}
\end{figure}

Ensuite, nous configurons les nodes DHT11 dans Node-RED pour séparer les données reçues entre Température et Humidité et les configurons pour actualiser les données toutes les 10 secondes.

\begin{figure}[H]
\centering
\includegraphics[width=12cm]{Images/Node-8.png}
\caption{Ajout d'un Inject Node pour répéter l'action chaque 10 secondes}
\label{Chap4.3.15}
\end{figure}

Dans la figure \ref{Chap4.3.15}, nous ajoutons un nœud Inject pour répéter l'action toutes les 10 secondes, permettant ainsi la mise à jour régulière des données.

\begin{figure}[H]
\centering
\includegraphics[width=15cm]{Images/Node-9.png}
\caption{Configurer un nœud Fonction pour séparer l'humidité de la température}
\label{Chap4.3.16}
\end{figure}

Dans la figure \ref{Chap4.3.16}, nous configurons un nœud Fonction pour séparer l'humidité de la température, ce qui permet d'obtenir des valeurs distinctes pour chaque paramètre.

\begin{figure}[H]
\centering
\includegraphics[width=15cm]{Images/Node-10.png}
\caption{Le résultat du nœud Debug}
\label{Chap4.3.17}
\end{figure}

La figure \ref{Chap4.3.17} montre le résultat obtenu à l'aide d'un nœud Debug, où les données séparées d'humidité et de température sont affichées.

De plus, nous créons des nœuds interrupteurs (Switches) pour contrôler les LEDs.


\begin{figure}[H]
\centering
\includegraphics[width=13cm]{Images/Node-6.png}
\caption{Nœud Switch}
\label{Chap4.3.18}
\end{figure}

Dans cette figure \ref{Chap4.3.18}, nous ajoutons un nœud Switch pour contrôler l'état des LEDs en fonction d'une condition spécifique.

\begin{figure}[H]
\centering
\includegraphics[width=12cm]{Images/Node-7.png}
\caption{Nœud Raspberry GPIO Output}
\label{Chap4.3.19}
\end{figure}

La figure \ref{Chap4.3.19} présente le nœud Raspberry GPIO Output utilisé pour contrôler les GPIO du Raspberry Pi et allumer/éteindre les LEDs.

Finalement, voici les nœuds et l'exécution finale de mon projet Node-RED.

\begin{figure}[H]
\centering
\includegraphics[width=15cm]{Images/Node-1.png}
\caption{Le projet en Node-RED}
\label{Chap4.3.20}
\end{figure}

La figure \ref{Chap4.3.20} présente le projet Node-RED complet avec tous les nœuds configurés et connectés.

\begin{figure}[H]
\centering
\includegraphics[width=6cm]{Images/Node-11.jpg} \includegraphics[width=6cm]{Images/Node-12.jpg}

\caption{Circuit de mon prototype miniature}
\label{Chap4.3.21}
\end{figure}

Dans la figure \ref{Chap4.3.21}, nous pouvons voir le circuit prototype miniature de notre projet, avec les composants connectés sur la breadboard et le Raspberry Pi.

\begin{figure}[H]
\centering
\includegraphics[height=15cm]{Images/1006591.jpg}
\caption{Le login du tableau de bord en mobile}
\label{Chap4.3.22}
\end{figure}

La figure \ref{Chap4.3.22} illustre l'interface de connexion (Dashboard Login) de notre projet, conçue pour une utilisation sur mobile. Cette figure démontre que notre projet dispose d'un accès réseau via l'adresse "192.168.3.192:1880". Ici, "192.168.3.192" correspond à l'adresse IP de notre Raspberry Pi, et "1880" indique le port sur lequel Node-RED est en cours d'exécution.

\begin{figure}[H]
\centering
\includegraphics[width=15cm]{Images/150131.png}
\includegraphics[width=15cm]{Images/150156.png}
\caption{Le tableau de bord en bureau}
\label{Chap4.3.23}
\end{figure}

La figure \ref{Chap4.3.23} montre le tableau de bord (Dashboard) de notre projet en bureau.

\subsection{Intégration dans le parc informatique}

Une fois notre application prête, nous intégrons celle-ci dans le parc informatique que nous avons créé dans le troisième chapitre à l'aide de GLPI.

Pour cela, nous installons le FusionInventory-Agent sur notre Raspberry Pi, comme illustré ci-dessous en figure \ref{fig:raspberry-fusion}.

\begin{figure}[H]
\centering
\includegraphics[width=15cm]{Images/raspberryfusion.png}
\caption{Installation de FusionInventory-Agent}
\label{fig:raspberry-fusion}
\end{figure}

Ensuite, nous configurons le FusionInventory-Agent pour qu'il envoie les données de notre périphérique au serveur GLPI à l'adresse "192.168.3.66 / glpi / fusioninventory ", comme présenté dans la figure \ref{fig:raspberry-fusion-config}.

\begin{figure}[H]
\centering
\includegraphics[width=15cm]{Images/raspberryfusion1.png}
\caption{Configuration de FusionInventory-Agent}
\label{fig:raspberry-fusion-config}
\end{figure}

Enfin, nous visualisons notre Raspberry Pi intégré dans le tableau de bord de GLPI \ref{fig:glpi-raspberry}.

\begin{figure}[H]
\centering
\includegraphics[width=16.5cm]{Images/RASPBERRYPIGLPI.png}
\caption{Tableau de bord de GLPI avec le Raspberry Pi intégré}
\label{fig:glpi-raspberry}
\end{figure}

\bigskip

\section{Conclusion}

En conclusion de ce chapitre dédié à l'intégration des objets connectés, nous avons exploré en détail les différentes étapes nécessaires pour incorporer ces dispositifs dans notre infrastructure existante. Les objets connectés, de par leur capacité à collecter, communiquer et analyser des données en temps réel, offrent un potentiel considérable pour améliorer l'efficacité et la qualité de vie dans divers domaines. \\

Nous avons commencé par définir ce qu'étaient les objets connectés, en mettant en évidence leur rôle essentiel dans la création de systèmes intelligents et autonomes. Ensuite, nous avons exploré les défis et opportunités liés à l'intégration de ces dispositifs au sein de notre entreprise. \\

Pour répondre à ces défis, nous avons présenté une solution basée sur l'utilisation de Raspberry Pi et de Node-RED. Cette combinaison nous permet de connecter divers dispositifs, tels que les lampes et le capteur de température et humidité, à notre réseau d'entreprise de manière cohérente et efficace. Nous avons également mis en évidence les étapes d'installation et de configuration de ces outils, ainsi que les mesures de sécurité mises en place pour protéger notre système. \\

Enfin, nous avons illustré la réalisation d'un projet concret en utilisant Node-RED pour contrôler des LEDs en fonction de données de capteur de température et d'humidité. Ce projet a été intégré dans notre parc informatique, ce qui nous permet de suivre et de gérer notre dispositif IoT à partir du système de gestion GLPI. \\

L'intégration des objets connectés représente un pas significatif vers une infrastructure informatique moderne et intelligente, et ce chapitre nous a fourni les connaissances et les outils nécessaires pour relever ce défi avec succès. \\