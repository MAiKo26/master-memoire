\chapter*{Conclusion Générale}
%\markboth{Conclusion Générale }{} %pour afficher l'entete
\addcontentsline{toc}{chapter}{Conclusion Générale }

{\fontsize{16pt}{20pt}\selectfont

Ce projet de Mémoire de Mastère a constitué une plongée approfondie dans l'univers des objets connectés au sein d'une entreprise. Au fil de ce travail, nous avons exploré trois aspects fondamentaux, piliers de toute organisation : le déploiement de l'infrastructure informatique, la mise en place du système d'information, et l'intégration des objets connectés. Chacun de ces aspects a fait l'objet d'une analyse exhaustive. \\

Lors de notre démarche pour le déploiement de l'infrastructure informatique, nous avons accordé une attention particulière à son rôle crucial dans l'amélioration des performances d'une entreprise. Nous avons établi un réseau MAN astucieusement conçu pour interconnecter les trois locaux de l'entreprise. L'usage de technologies de pointe, notamment le TP-Link CPE610, a permis d'établir des liaisons haut débit, résistantes aux interférences, assurant ainsi une connectivité stable et fiable entre les sites distants. \\

Le déploiement du système d'information, second pilier de notre travail, s'est déroulé avec une rigueur méthodique. Le choix judicieux de solutions telles que Ubuntu Server 22.04, KVM et GLPI a abouti à la création d'un système d'information sur mesure, parfaitement adapté aux besoins spécifiques de l'entreprise. Les impératifs de sécurité du réseau, de performances optimales, et de compatibilité logicielle sans faille ont constamment guidé nos décisions. Chaque étape de déploiement a été soigneusement décortiquée, exposant clairement les motivations stratégiques derrière nos choix techniques. \\

En ce qui concerne l'intégration des objets connectés, le troisième volet de notre travail, nous avons ouvert une porte sur l'avenir des entreprises informatisées. Des dispositifs variés tels que des capteurs, des lampes et des machines de pointage ont été habilement connectés au réseau de l'entreprise. L'utilisation astucieuse de la carte IoT Raspberry Pi, combinée à des logiciels spécialisés, a permis de contrôler ces objets à distance tout en collectant en temps réel des données cruciales. Cette section a mis en évidence les nouvelles perspectives qu'offre l'Internet des Objets (IoT) et comment ces technologies transforment le fonctionnement des entreprises. \\

En conclusion, cette Mémoire de Mastère témoigne que l'intégration des objets connectés dépasse largement le cadre d'une simple tendance technologique. Elle représente une transformation profonde dans la manière dont les entreprises évoluent et interagissent avec leur environnement. Nous sommes fermement convaincus que les entreprises qui sauront capitaliser sur ces avancées technologiques seront mieux préparées pour faire face aux défis de demain et maintenir leur compétitivité dans un monde en constante évolution. \\




}