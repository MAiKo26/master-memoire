\chapter*{Conclusion Générale}
%\markboth{Conclusion Générale }{} %pour afficher l'entete
\addcontentsline{toc}{chapter}{Conclusion Générale }

Ce projet de Mémoire de Mastère a exploré en profondeur l'univers de l'informatique en entreprise. Nous avons exploré trois aspects fondamentaux : le déploiement de l'infrastructure informatique, la mise en place du système d'information et l'intégration des objets connectés.

Lors du déploiement de l'infrastructure informatique, nous avons conçu un réseau MAN pour relier les trois locaux de l'entreprise, utilisant des technologies avancées comme le TP-Link CPE610 pour garantir une connectivité stable.

Le déploiement du système d'information a été rigoureux, avec des choix technologiques judicieux comme Ubuntu Server 22.04, KVM et GLPI pour créer un système sur mesure répondant aux besoins spécifiques de l'entreprise.

Pour l'intégration des objets connectés, divers dispositifs, tels que des capteurs, des lampes et des machines de pointage, ont été connectés au réseau à l'aide de la carte Raspberry Pi et de logiciels spécialisés.

En conclusion, cette Mémoire de Mastère témoigne que l'intégration des objets connectés dépasse largement le cadre d'une simple tendance technologique. Elle représente une transformation profonde dans la manière dont les entreprises évoluent et interagissent avec leur environnement. Nous sommes fermement convaincus que les entreprises qui sauront capitaliser sur ces avancées technologiques seront mieux préparées pour faire face aux défis de demain et maintenir leur compétitivité dans un monde en constante évolution.




