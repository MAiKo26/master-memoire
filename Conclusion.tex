\chapter{Conclusion Générale }
%\markboth{Conclusion Générale }{} %pour afficher l'entete
%\addcontentsline{toc}{chapter}{Conclusion Générale }



En conclusion, ce projet de Mémoire De Mastère s'est focalisé sur trois aspects essentiels d'une entreprise système : le déploiement de l'infrastructure informatique, le déploiement du système d'information et l'intégration des objets connectés. Chacune de ces parties a été abordée de manière approfondie, en mettant l'accent sur l'analyse des besoins de l'entreprise et la sélection rigoureuse des outils et technologies appropriés. \\

Dans le cadre du déploiement de l'infrastructure informatique, une attention particulière a été accordée à l'implémentation d'un réseau MAN reliant les trois bureaux de l'entreprise. Grâce à l'utilisation du TP-Link CPE610 et à l'utilisation de technologies de pointe, nous avons établi des liaisons haut débit et résistantes aux interférences, assurant ainsi une connectivité stable et fiable entre les différents sites. \\

Le déploiement du système d'information a été réalisé en choisissant avec soin les éléments constitutifs du SI. Des solutions telles que le pare-feu pfSense, Ubuntu Server 22.04 et Odoo ont été installées et configurées pour répondre aux besoins spécifiques de l'entreprise, en garantissant la sécurité du réseau, la performance et la compatibilité logicielle. \\

L'intégration des objets connectés au système d'information a permis de connecter des dispositifs tels que des capteurs, des lampes et des machines de pointage au réseau de l'entreprise. Des cartes IoT telles que Raspberry Pi et Arduino, ainsi que des logiciels spécialisés, ont été utilisés pour contrôler ces objets à distance et collecter des données en temps réel. \\

En combinant ces trois volets, nous avons créé une infrastructure solide, sécurisée et adaptée aux besoins opérationnels de l'entreprise. Ce rapport peut servir de référence précieuse pour les entreprises souhaitant entreprendre des projets similaires, tout en sensibilisant les étudiants à l'importance de ces domaines en constante évolution. L'intégration des objets connectés et l'optimisation du système d'information offrent des avantages significatifs en termes d'efficacité opérationnelle, de prise de décision et de compétitivité sur le marché. \\